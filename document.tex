\documentclass[12pt]{article}
\usepackage{amsmath}
\usepackage{amssymb}
\usepackage{mathrsfs}
\usepackage[russian]{babel}
\voffset=-10mm
\oddsidemargin=5mm
\evensidemargin=0mm
\textheight=235mm
\textwidth=170mm
\topmargin=-7.2mm
\newtheorem{theorem}{\hskip\parindent Теорема}%[section]
\newtheorem{definition}{\hskip\parindent Определение}%[section]
\newtheorem{corollary}{\hskip\parindent Следствие}%[section]
\newtheorem{lemma}{\hskip\parindent Лемма}%[section]
\newtheorem{remark}{\hskip\parindent Замечание}%[section]
\newtheorem{example}{\hskip\parindent Пример}%[section]
\begin{document}
\section{Линейные ограниченные функционалы на нормированном пространстве}
Пусть $(X,\|\cdot\|)$ --- нормированное пространство над полем $K$.
\begin{definition}
	\textbf{Функционалом} на $X$ будем называть отображение $f: X\to K$.
\end{definition}
\begin{definition}
	Функционал $f$ называется \textbf{линейным}, если для любых $\alpha$, $\beta\in K$ и любых $x$, $y\in X$
	\begin{equation*}
		f(\alpha x+\beta y)=\alpha\cdot f(x)+\beta\cdot f(y)
	\end{equation*}
\end{definition}
\begin{definition}
	Функционал $f$ называется \textbf{ограниченным}, если существует постоянная $C\geqslant0$ такая, что для любого $x\in X$
	\begin{equation*}
		|f(x)|\leqslant C\|x\|
	\end{equation*}
\end{definition}

Пусть $(X,\|\cdot\|_X)$, $(Y,\|\cdot\|_Y)$ --- нормированные пространства. Обозначим $LB(X,Y)$ --- множество линейных ограниченных операторов $A:X\to Y$.
\begin{theorem}
$LB(X,Y)$ полно, если $Y$ полно.
\end{theorem}

Множество $LB(X,K)$ линейных ограниченных функционалов называется \textbf{сопряжённым пространством} к $X$ и обозначается $X'$.

Так как в курсе $K=\mathbb{R}$ или $K=\mathbb{C}$ и оба эти пространства полны, то $X'$ полно.
\section{Норма функционала}
\begin{definition}
	Пусть $X$ --- нормированное пространство, $f\in X'$. \textbf{Норма} функционала $f$:
	\begin{equation*}
		\|f\|=\sup\limits_{\|x\|\leqslant1}|f(x)|
	\end{equation*}
\end{definition}
\begin{theorem}

\end{theorem}
\section{Примеры вычисления норм функционалов}
\begin{example}
	\begin{equation*}
		X=\mathbb{R}^n
	\end{equation*}
\end{example}
Зафиксируем произвольный базис $\left\{e_k\right\}_{k=1}^{n}$, тогда любой элемент $x\in X$ представляется в виде
\begin{equation*}
	x=\sum\limits_{k=1}^{n}\alpha_k e_k,
\end{equation*}
Будем считать, что норма $\|\cdot\|$ --- евклидова. Пусть $f\in X'$, тогда
\begin{equation*}
	f(x)=f\left(\sum\limits_{k=1}^{n}\alpha_k e_k\right)=\sum\limits_{k=1}^{n}\alpha_k f(e_k)
\end{equation*}
Обозначим $\xi_k=f(e_k)$, $\xi=(\xi_1,\dots, x_n)$, тогда
\begin{equation*}
	f(x)=\langle x,\xi\rangle
\end{equation*}
Используя неравенство Коши-Буняковского, имеем
\begin{equation*}
	|f(x)|\leqslant\|x\|\cdot\|\xi\| \Longrightarrow \|f\|\leqslant\|\xi\|
\end{equation*}
Положим $x_0=\frac{\xi}{\|\xi\|}\in\mathbb{R}^n$, тогда
\begin{equation*}
	|f(x_0)|=\left|\langle\frac{\xi}{\|\xi\|},\xi\rangle\right|=\frac{\|\xi\|^2}{\|\xi\|}=\|\xi\| \Longrightarrow \|f\|\geqslant\xi
\end{equation*}
Таким образом, $\|f\|=\|\xi\|$.
\begin{example}
\begin{equation*}
	X=l^p, \quad 1\leqslant p<\infty
\end{equation*}
\end{example}
Пусть $u\in l^q$, 
\begin{equation*}
f(x)=\sum\limits_{k=1}^{\infty}x_ku_k
\end{equation*}
Неравенство Гёльдера
\begin{equation*}
|f(x)|=\left|\sum\limits_{k=1}^{\infty}x_ku_k\right|\leqslant\left(\sum\limits_{k=1}^{\infty}|x_k|^p\right)^{\frac{1}{p}}\cdot\left(\sum\limits_{k=1}^{\infty}|u_k|^q\right)^{\frac{1}{q}}=\|x\|_p\|u\|_q \Longrightarrow \|f\|\leqslant\|u\|_q
\end{equation*}
Обозначим
\begin{equation*}
e_k=(0,0,\dots,0,1,0,0,\dots),
\end{equation*}
тогда любой элемент $x\in l^p$ представляется в виде
\begin{equation*}
x=\sum\limits_{k=1}^{\infty}x_ke_k
\end{equation*}
так как
\begin{equation*}
\left\|x-\sum\limits_{k=1}^{n}x_ke_k\right\|=\left(\sum\limits_{k=n+1}^{\infty}|x_k|^p\right)^{\frac{1}{p}}\to0, \quad n\to\infty,
\end{equation*}
поэтому
\begin{equation*}
f(x)=f\left(\sum\limits_{k=1}^{\infty}x_ke_k\right)=\sum\limits_{k=1}^{\infty}x_kf(e_k)=\sum\limits_{k=1}^{\infty}x_ku_k
\end{equation*}
Положим 
\begin{equation*}
x_0^{(n)}=\left(|u_1|^{q-1}\cdot\operatorname{sgn}u_1,\dots,|u_n|^{q-1}\cdot\operatorname{sgn}u_n,0,0,\dots\right)
\end{equation*}
\begin{equation*}
\left|f(x_0^{(n)})\right|=\sum\limits_{k=1}^{n}|u_k|^q\leqslant\left(\sum\limits_{k=1}^{n}|u_k|^{p\cdot(q-1)}\right)^{\frac{1}{p}}\cdot\|f\|=\left(\sum\limits_{k=1}^{n}|u_k|^{q}\right)^{\frac{1}{p}}\cdot\|f\|
\end{equation*}
\begin{equation*}
	\sum\limits_{k=1}^{n}|u_k|^q\leqslant\left(\sum\limits_{k=1}^{n}|u_k|^{q}\right)^{\frac{1}{p}}\cdot\|f\|
\end{equation*}
\begin{equation*}
	\left(\sum\limits_{k=1}^{n}|u_k|^{q}\right)^{1-\frac{1}{p}}\leqslant\|f\|
\end{equation*}
\begin{equation*}
	\left(\sum\limits_{k=1}^{n}|u_k|^{q}\right)^{\frac{1}{q}}\leqslant\|f\| \quad \forall\, n\in\mathbb{N}
\end{equation*}
Следовательно, ряд $\sum\limits_{k=1}^{\infty}|u_k|^{q}$ сходится и при этом $\|u\|\leqslant\|f\|$. Учитывая неравенство Гёльдера, имеем $\|f\|=\|u\|_q$.
\begin{example}
	\begin{equation*}
		X=L^p[0,1], \quad f(x)=\int\limits_{0}^{1}a(t)x(t)\,dt, \quad a\in L^q[0,1], \quad \frac{1}{p}+\frac{1}{q}=1
	\end{equation*}
\end{example}
Линейность очевидна.
\begin{equation*}
	\left|\int\limits_{0}^{1}a(t)x(t)\,dt\right|\leqslant\left(\int\limits_{0}^{1}|x(t)|^p\,dt\right)^{\frac{1}{p}}\cdot\left(\int\limits_{0}^{1}|a(t)|^q\,dt\right)^{\frac{1}{q}}=\|x\|_p\cdot\|a\|_q
\end{equation*}
то есть $|f(x)|\leqslant \|a\|_q\cdot \|x\|_p$.

$\|f\|=\|a\|_q$.
\begin{example}
	\begin{equation*}
		X=C[0,1],
	\end{equation*}
\end{example}
\begin{theorem}[Рисс]
	Для любого $f\in(C[0,1])'$ существует функция $g\in BV[0,1]$ ограниченной вариации, непрерывная слева, $g(0)=0$, такая, что
	\begin{equation*}
		f(x)=\int\limits_{0}^{1}x(t)\,dg(t)
	\end{equation*}
	При этом $\|f\|=\operatorname{Var}\limits_{[0,1]}(g)$
\end{theorem}
\end{document}