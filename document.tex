\documentclass[12pt]{article}
\usepackage{amsmath}
\usepackage{amssymb}
\usepackage{mathrsfs}
\usepackage[russian]{babel}
\voffset=-10mm
\oddsidemargin=5mm
\evensidemargin=0mm
\textheight=235mm
\textwidth=170mm
\topmargin=-7.2mm
\newtheorem{theorem}{\hskip\parindent Теорема}%[section]
\newtheorem{definition}{\hskip\parindent Определение}%[section]
\newtheorem{corollary}{\hskip\parindent Следствие}%[section]
\newtheorem{lemma}{\hskip\parindent Лемма}%[section]
\newtheorem{remark}{\hskip\parindent Замечание}%[section]
\newtheorem{example}{\hskip\parindent Пример}%[section]
\begin{document}
	\section{Линейные ограниченные функционалы на нормированном пространстве}
	Пусть $(X,\|\cdot\|)$ --- нормированное пространство над полем $K$.
	\begin{definition}
		\textbf{Функционалом} на $X$ будем называть отображение $f: X\to K$.
	\end{definition}
	\begin{definition}
		Функционал $f$ называется \textbf{линейным}, если для любых $\alpha$, $\beta\in K$ и любых $x$, $y\in X$
		\begin{equation*}
			f(\alpha x+\beta y)=\alpha\cdot f(x)+\beta\cdot f(y)
		\end{equation*}
	\end{definition}
	\begin{definition}
		Функционал $f$ называется \textbf{ограниченным}, если существует постоянная $C\geqslant0$ такая, что для любого $x\in X$
		\begin{equation*}
			|f(x)|\leqslant C\|x\|
		\end{equation*}
	\end{definition}

Пусть $(X,\|\cdot\|_X)$, $(Y,\|\cdot\|_Y)$ --- нормированные пространства. Обозначим $LB(X,Y)$ --- множество линейных ограниченных операторов $A:X\to Y$.
\begin{theorem}
	$LB(X,Y)$ полно, если $Y$ полно.
\end{theorem}

Множество $LB(X,K)$ линейных ограниченных функционалов называется \textbf{сопряжённым пространством} к $X$ и обозначается $X'$.

Так как в курсе $K=\mathbb{R}$ или $K=\mathbb{C}$ и оба эти пространства полны, то $X'$ полно.
	\section{Норма функционала}
	\begin{definition}
		Пусть $X$ --- нормированное пространство, $f\in X'$. \textbf{Норма} функционала $f$:
		\begin{equation*}
			\|f\|=\sup\limits_{\|x\|\leqslant1}|f(x)|
		\end{equation*}
	\end{definition}
\begin{theorem}
	
\end{theorem}
	\section{Примеры вычисления норм функционалов}
	\begin{example}
		\begin{equation*}
			X=\mathbb{R}^n
		\end{equation*}
	\end{example}
	Зафиксируем произвольный базис $\left\{e_k\right\}_{k=1}^{n}$, тогда любой элемент $x\in X$ представляется в виде
	\begin{equation*}
		x=\sum\limits_{k=1}^{n}\alpha_k e_k,
	\end{equation*}
	Будем считать, что норма $\|\cdot\|$ --- евклидова. Пусть $f\in X'$, тогда
	\begin{equation*}
		f(x)=f\left(\sum\limits_{k=1}^{n}\alpha_k e_k\right)=\sum\limits_{k=1}^{n}\alpha_k f(e_k)
	\end{equation*}
	Обозначим $\xi_k=f(e_k)$, $\xi=(\xi_1,\dots, x_n)$, тогда
	\begin{equation*}
		f(x)=\langle x,\xi\rangle
	\end{equation*}
	Используя неравенство Коши-Буняковского, имеем
	\begin{equation*}
		|f(x)|\leqslant\|x\|\cdot\|\xi\| \Longrightarrow \|f\|\leqslant\|\xi\|
	\end{equation*}
	Положим $x=\frac{\xi}{\|\xi\|}$, тогда
	\begin{equation*}
		|f(x)|=\left|\langle\frac{\xi}{\|\xi\|},\xi\rangle\right|=\frac{\|\xi\|^2}{\|\xi\|}=\|\xi\| \Longrightarrow \|f\|\geqslant\xi
	\end{equation*}
	Таким образом, $\|f\|=\|\xi\|$.
	\begin{example}
		\begin{equation*}
			X=L^p[0,1], \quad f(x)=\int\limits_{0}^{1}a(t)x(t)\,dt, \quad a\in L^q[0,1], \quad \frac{1}{p}+\frac{1}{q}=1
		\end{equation*}
	\end{example}
Линейность очевидна.
\begin{equation*}
	\left|\int\limits_{0}^{1}a(t)x(t)\,dt\right|\leqslant\left(\int\limits_{0}^{1}|x(t)|^p\,dt\right)^{\frac{1}{p}}\cdot\left(\int\limits_{0}^{1}|a(t)|^q\,dt\right)^{\frac{1}{q}}=\|x\|_p\cdot\|a\|_q
\end{equation*}
то есть $|f(x)|\leqslant \|a\|_q\cdot \|x\|_p$.

Положим
\begin{equation*}
	x(t)=\frac{f}{f},
\end{equation*}
тогда
\begin{equation*}
	|f(x)|=
\end{equation*}
Таким образом, $\|f\|=\|a\|_q$.
\begin{example}
	\begin{equation*}
		X=l^p, \quad 1\leqslant p<\infty
	\end{equation*}
\end{example}
\end{document}