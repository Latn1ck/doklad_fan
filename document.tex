\documentclass[12pt]{article}
\usepackage{amsmath}
\usepackage{amssymb}
\usepackage{mathrsfs}
\usepackage[russian]{babel}
\voffset=-10mm
\oddsidemargin=5mm
\evensidemargin=0mm
\textheight=235mm
\textwidth=170mm
\topmargin=-7.2mm
\newtheorem{theorem}{\hskip\parindent Теорема}%[section]
\newtheorem{definition}{\hskip\parindent Определение}%[section]
\newtheorem{corollary}{\hskip\parindent Следствие}%[section]
\newtheorem{lemma}{\hskip\parindent Лемма}%[section]
\newtheorem{remark}{\hskip\parindent Замечание}%[section]
\newtheorem{example}{\hskip\parindent Пример}%[section]
\begin{document}
	\section{Линейные ограниченные функционалы на нормированном пространстве}
	Пусть $(X,\|\cdot\|)$ --- нормированное пространство над полем $K$.
	\begin{definition}
		\textbf{Функционалом} на $X$ будем называть отображение $f: X\to K$.
	\end{definition}
	\begin{definition}
		Функционал $f$ называется \textbf{линейным}, если для любых $\alpha$, $\beta\in K$ и любых $x$, $y\in X$
		\begin{equation*}
			f(\alpha x+\beta y)=\alpha\cdot f(x)+\beta\cdot f(y)
		\end{equation*}
	\end{definition}
	\begin{definition}
		Функционал $f$ называется \textbf{ограниченным}, если существует постоянная $C\geqslant0$ такая, что для любого $x\in X$
		\begin{equation*}
			|f(x)|\leqslant C\|x\|
		\end{equation*}
	\end{definition}
	\section{Норма функционала}
	\section{Примеры вычисления норм функционалов}
\end{document}